\documentclass[12pt,leqno, twoside]{mwart}
\usepackage[utf8]{inputenc}
\usepackage{polski}
\usepackage{a4wide}
%-------------------------------------------------------------------------
%paginy!
\usepackage{fancyhdr}
\pagestyle{fancy}
\fancyhead{}
\renewcommand{\sectionmark}[1]{\markright{\thesection.\ #1 }}
\fancyhead[LE]{Podręczna kucharyna. Harmonogram}
\fancyhead[RO]{\rightmark}
\fancyfoot{} % clear all footer fields
\fancyfoot[LE,RO]{}
\fancyfoot[CE]{\thepage}
\fancyfoot[CO]{\thepage}
\addtolength{\headheight}{1.5pt} % pionowy odstep na kreske
\renewcommand{\headrulewidth}{0.4pt}
\renewcommand{\footrulewidth}{0.0pt}
%-------------------------------------------------------------------------
\makeatletter
\renewcommand{\@biblabel}[1]{\quad #1.}
\makeatother
%-------------------------------------------------------------------------
\renewcommand{\figurename}{Rys.}
\renewcommand{\labelitemi}{-}

\def\tm{\leavevmode\hbox{$\rm {}^{TM}$}}

\begin{document}

\thispagestyle{empty}
\begin{center}
Licencjacka Pracownia Oprogramowania \\ Instytut
Informatyki Uniwersytetu Wrocławskiego \\
\vspace{4cm}
\Large Agnieszka Góralczyk, Adrian Dyliński, Dominik Bruliński \\
\vspace{0.5cm}
\huge Dokumentacja projektu\\ \textbf{Podręczna kucharyna}\\ \Large Harmonogram\\
\vspace{1cm}
\normalsize \today \\
\vspace{2cm}
\normalsize Wersja 0.02
\end{center}

\newpage


\begin{table}
	\centering
	\caption{Historia zmian w dokumencie}
		\begin{tabular}{|r|c|c|l|l|}
		\hline
		Lp. & Data       & Nr wersji & Autor               & Zmiana \\ \hline
		1.   & 2012-05-05 & 0.01 & Adrian Dyliński, Aron Mojsak & Utworzenie dokumentu \\ \hline
		2.  & 2012-05-20 & 0.02 & Dominik Bruliński & Uzupełnienie treści \\ \hline
		\end{tabular}
\end{table}

\newpage

\tableofcontents
\newpage

\section{Wprowadzenie}
Dokument przedstawia harmonogram prac nad projektem \emph{Podręczna kucharyna}.

\section{Harmonogram prac}

Daty przedstawione poniżej prezentują końcowe terminy wykonania poszczególnych prac.


\begin{table}
	\centering
		\begin{tabular}{|p{4cm}|p{11cm}|}
		\hline
		\bf{Data}            & \bf{Zadanie} \\ \hline
		5 marca 2012 r. & określenie wymagań i przygotowanie dokumentów: przypadki użycia, wykaz dokumentów \\ \hline
		
		18 marca 2012 r. & przygotowanie analizy rozwiązania i standardów pracy\\ \hline
		
		16 kwietnia 2012 r. & sporządzenie planu testowania \\ \hline
		
		19 kwietnia 2012 r. & wykonanie poprawek i ostateczna weryfikacja dokumentacji przedsięwzięcia \\ \hline
		
		20 kwietnia 2012 r. & przygotowanie środowiska programistycznego (uruchomienie serwera testowego i~systemu zarządzania rozwojem oprogramowania), przypisanie zadań poszczególnym osobom zaangażowanym w realizację projektu \\ \hline
		25 kwietnia 2012 r. & umożliwienie komunikacji z bazą danych \\ \hline
		
		26 kwietnia 2012 r. & zaprojektowanie interfejsu użytkownika  \\ \hline
		
		30 kwietnia 2012 r. & zaprogramowanie fragmentu interfejsu użytkownika pozwalającego na zarządzanie przepisami \\ \hline
		
		7 maja 2012 r. & implementacja komunikacji między modułami projektu \\ \hline
		
		12 maja 2012 r. & umożliwienie dodawania, zmiany i usuwania przepisów \\ \hline
		
		14 maja 2012 r. & umożliwienie przeglądania przepisów \\ \hline
		
		27 maja 2012 r. & umożliwienie zarządzania składnikami i dostosowanie interfejsu użytkownika\\ \hline
		
		3 czerwca 2012 r. & umożliwienie nadawania etykiet posiłkom \\ \hline
		
		10 czerwca 2012 r. & umożliwienie wyszukiwania przepisów za pomocą etykiet \\ \hline
		
		15 czerwca 2012 r. & implementacja automatycznego etykietowania przepisów według użytych składników \\ \hline
		
		22 czerwca 2012 r. & testowanie aplikacji i~poprawianie błędów \\ \hline
		
		24 czerwca 2012 r. & udostępnienie aplikacji użytkownikom do pobrania ze strony projektu \\ \hline
		
		1 lipca 2012 r. & umożliwienie określania wartości kalorycznych posiłków na podstawie użytych składników \\ \hline

		10 lipca 2012 r. & możliwość obliczania wartości odżywczych niezależnie od użytych miar wagi lub objętości \\ \hline
		
		27 lipca 2012 r. & możliwość dodawania zdjęć składników i posiłków \\ \hline
		
		10 sierpnia 2012 r. & możliwość zakładania kont użytkowników \\ \hline
		
		25 sierpnia 2012 r. & możliwość zarządzania uprawnieniami użytkowników \\ \hline
		
		30 sierpnia 2012 r. & możliwość komentowania przepisów \\ \hline
		
		10 września 2012 r. & testowanie aplikacji \\ \hline
		
		15 września 2012 r. & udostępnie aplikacji ograniczonej liczbie użytkowników w celu testów \\ \hline
		
		29 września 2012 r. & usunięcie błędów i wprowadzenie poprawek \\ \hline
		
		30 września 2012 r. & udostępnienie aplikacji za pośrednictwem Google Play\tm \\ \hline
		
		
		\end{tabular}
\end{table}


\end{document}
