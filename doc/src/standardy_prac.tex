\documentclass[12pt,leqno, twoside]{mwart}
\usepackage[utf8]{inputenc}
\usepackage[polish]{babel}
\usepackage{polski}
\usepackage{a4wide}
%-------------------------------------------------------------------------
%paginy!
\usepackage{fancyhdr}
\pagestyle{fancy}
\fancyhead{}
\renewcommand{\sectionmark}[1]{\markright{\thesection.\ #1 }}
\fancyhead[LE]{Podręczna kucharyna. Standardy prac}
\fancyhead[RO]{\rightmark}
\fancyfoot{} % clear all footer fields
\fancyfoot[LE,RO]{}
\fancyfoot[CE]{\thepage}
\fancyfoot[CO]{\thepage}
\addtolength{\headheight}{1.5pt} % pionowy odstep na kreske
\renewcommand{\headrulewidth}{0.4pt}
\renewcommand{\footrulewidth}{0.0pt}
%-------------------------------------------------------------------------
\makeatletter
\renewcommand{\@biblabel}[1]{\quad #1.}
\makeatother
%-------------------------------------------------------------------------
\renewcommand{\figurename}{Rys.}
\renewcommand{\labelitemi}{-}

\def\tm{\leavevmode\hbox{$\rm {}^{TM}$}}

\begin{document}

\thispagestyle{empty}
\begin{center}
Licencjacka Pracownia Oprogramowania \\ Instytut
Informatyki Uniwersytetu Wrocławskiego \\
\vspace{4cm}
\Large Agnieszka Góralczyk, Adrian Dyliński, Dominik Bruliński \\
\vspace{0.5cm}
\huge Dokumentacja projektu\\ \textbf{Podręczna kucharyna}\\ \Large Standardy prac\\
\vspace{1cm}
\normalsize \today \\
\vspace{2cm}
\normalsize Wersja 0.02
\end{center}

\newpage


\begin{table}
	\centering
	\caption{Historia zmian w dokumencie}
		\begin{tabular}{|r|c|c|l|l|}
		\hline
		Lp. & Data       & Nr wersji & Autor               & Zmiana \\ \hline
		1.  & 2012-04-15 & 0.01 & Dominik Bruliński & Utworzenie dokumentu \\ \hline
		2.  & 2012-04-16 & 0.02 & Adrian Dyliński & Poprawki językowe \\ \hline
		\end{tabular}
\end{table}

\newpage

\tableofcontents
\newpage

\section{Wstęp}
	W niniejszym dokumencie wyszczególnione są wymagania odnoszące się do standardów prac
	nad aplikacją \emph{Podręczna kucharyna}.

	\section{Standardy dotyczące kodu źródłowego}
	
	Kod źródłówy systemu powinien spełniac następujące wymagania:
	\begin{itemize}
		\item języki programowania: Java, XML
		\item nazewnictwo:
			\begin{itemize}
				\item wszystkie jednostki składniowe nazwane w języku angielskim,
				\item klasy i metody publiczne: styl ,,wielbłądzi"\footnote{Wszystkie poza pierwszym słowa składowe nazwy
					rozpoczyna się wielką literą, bez użycia odstępów, np. \emph{mojaNazwaMetody}.},
				\item zmienne: pisane małymi literami,
			\end{itemize}
 		\item rozmiar tabulacji: 4 znaki odstępu,
		\item maksymalna długość wiersza: 78 znaków,
		\item komentarze w kodzie źródłowym:
			\begin{itemize}
				\item każdy moduł systemu zawiera krótki opis,
				\item każda definicja klasy zawiera krótki opis,
				\item przy definicji metody opisane znaczenie argumentów wejściowych i rezultatu działania,
				\item w ciele procedur/funkcji komentarze wymagane tylko przy skomplikowanych fragmentach kodu.
			\end{itemize}
	\end{itemize}

	\section{Standardy dotyczące narzędzi}
	
	Podczas prac nad systemem jako obowiązkowe uznaje się używanie następujących narzędzi:
	\begin{itemize}
		\item system kontroli wersji -- zmiany powinny być wprowadzane do systemu na koniec dnia roboczego lub po zakończeniu prac nad pewnym zadaniem,
		\item system dysponowania zadań -- system musi umożliwiać opisywanie zadań, przyporządkowywanie ich do poszczególnych osób i rejestrację czasu pracy; podjęcie przez programistę prac nad pewnym zadaniem powinno mieć odzwierciedlenie w systemie,
		\item system śledzenia błędów -- błędy znalezione podczas testowania powinny być szczegółowo dokumentowane.
	\end{itemize}

	Ze względu na niższe koszty, zaleca się używanie darmowych narzędzi, realizujących powyższe wymagania.
	
\end{document}