\documentclass[12pt,leqno, twoside]{mwart}
\usepackage[utf8]{inputenc}
\usepackage[polish]{babel}
\usepackage{polski}
\usepackage{a4wide}
\usepackage{url}
\usepackage{hyperref}
%-------------------------------------------------------------------------
%paginy!
\usepackage{fancyhdr}
\pagestyle{fancy}
\fancyhead{}
\renewcommand{\sectionmark}[1]{\markright{\thesection.\ #1 }}
\fancyhead[LE]{Podręczna kucharyna. Warunki korzystania z programu i oprogramowanie firm trzecich}
\fancyhead[RO]{\rightmark}
\fancyfoot{} % clear all footer fields
\fancyfoot[LE,RO]{}
\fancyfoot[CE]{\thepage}
\fancyfoot[CO]{\thepage}
\addtolength{\headheight}{1.5pt} % pionowy odstep na kreske
\renewcommand{\headrulewidth}{0.4pt}
\renewcommand{\footrulewidth}{0.0pt}
%-------------------------------------------------------------------------
\makeatletter
\renewcommand{\@biblabel}[1]{\quad #1.}
\makeatother
%-------------------------------------------------------------------------
\renewcommand{\figurename}{Rys.}
\renewcommand{\labelitemi}{-}
\def\tm{\leavevmode\hbox{$\rm {}^{TM}$}}

\begin{document}

\thispagestyle{empty}
\begin{center}
Licencjacka Pracownia Oprogramowania \\ Instytut
Informatyki Uniwersytetu Wrocławskiego \\
\vspace{4cm}
\Large Agnieszka Góralczyk, Adrian Dyliński, Dominik Bruliński \\
\vspace{0.5cm}
\huge Dokumentacja projektu\\ \emph{Podręczna kucharyna}\\ \Large Warunki korzystania z programu i oprogramowanie firm trzecich użyte w aplikacji \emph{Podręczna kucharyna} \\
\vspace{1cm}
\normalsize \today \\
\vspace{2cm}
\normalsize Wersja 0.02
\end{center}

\newpage


%trzeba zmieniać historię zmian, strone tytułowa i \fancyhead :)


\begin{table}
	\centering
	\caption{Historia zmian w dokumencie}
		\begin{tabular}{|r|c|c|l|l|}
		\hline
		Lp. & Data       & Nr wersji & Autor               & Zmiana \\ \hline
		1.   & 2012-06-15 & 0.01 & Adrian Dyliński& Utworzenie dokumentu, dodanie licencji na oprogramowanie firm trzecich\\ \hline
		2. & 2012-06-20 & 0.02 & Adrian Dyliński & Dodanie sekcji ,,Warunki korzystania z programu'' i ,,Polityka prywatności''\\ \hline
		\end{tabular}
\end{table}

\newpage

\tableofcontents
\newpage
\section{Warunki korzystania z aplikacji \emph{Podręczna kucharyna} }
Aplikacja \emph{Podręczna kucharyna} wydana jest na licencji GNU Lesser General Public License w wersji 3.0.
Oznacza to, że kod źródłowy jest otwarty (ogólnodostępny) i można go modyfikować oraz używać w swoich aplikacjach bez ograniczeń. \\
Szczegółowe warunki licencji dostępne są pod adresem \url{"http://www.gnu.org/copyleft/lgpl"}. \\

Dokumentacja objęta jest licjencją GNU Free Documentation License. \\
Znajduje się ona pod adresem \url{"http://www.gnu.org/licenses/fdl-1.3.html"}.
\section{Polityka prywatności}

\subsection{Dane osobowe}
1. Administratorem danych osobowych udostępnionych przez Użytkownika jest 
Adrian Dyliński zameldowany w Szczecinie przy B. Śmiałego 32/6. Adrian Dyliński przetwarza dane osobowe w sposób zgodny z obowiązującymi przepisami prawa, w szczególności dotyczących odpowiedniego zabezpieczenia danych. \\
2. Adrian Dyliński przetwarza dane osobowe Użytkownika w zakresie niezbędnym do nawiązania, ukształtowania treści, zmiany, rozwiązania oraz prawidłowej realizacji usług świadczonych drogą elektroniczną. \\
3. Zbiór danych osobowych został zarejestrowany w biurze Generalnego Inspektora Ochrony Danych Osobowych. \\
4. Użytkownik Portalu ma prawo wglądu w swoje dane oraz prawo ich poprawiania. Podanie danych osobowych przez Użytkownika Aplikacji jest dobrowolne. \\
5. Adrian Dyliński nie sprzedaje i nie udostępnia danych osobowych Użytkowników Aplikacji osobom lub podmiotom trzecim, a dostęp do nich posiada jedynie ograniczone grono osób upoważnionych.  \\
6. Podczas rejestracji Użytkownik Aplikacji może wyrazić dodatkową zgodę na podstawie Ustawy z dnia 18 lipca 2002 r. o świadczeniu usług drogą elektroniczną (Dz. U. z 2002 r., Nr 144, poz. 1204) na otrzymywanie na swój adres elektroniczny informacji o produktach oraz usługach dotyczących oferty związanej z prowadzeniem serwisu e-psychologowie.pl. \\
7. Podane przez Użytkownika Aplikacji dane osobowe mogą być przetwarzane dla celów marketingowych jedynie przez Adrian Dyliński i podmioty przez nią upoważnione na podstawie przepisów ustawy z dnia 29 sierpnia 1997 r. o ochronie danych osobowych. \\
8. Adrian Dyliński zapewnia Użytkownikom Aplikacji możliwość usunięcia swoich danych z bazy danych.  \\
9. W przypadku jakichkolwiek pytań dotyczących zasad zachowania poufności, praktyk stosowanych w Serwisie bądź korzystania z usług, prosimy o kontakt pod adresem biuro@newsgastro.pl

\subsection{Przechowywanie i ochrona informacji}

1. Dane osobowe Użytkowników Aplikacji są przechowywane w bazie danych, w której zastosowano środki techniczne i organizacyjne zapewniające ochronę przetwarzanych danych zgodne z wymaganiami określonymi w przepisach o ochronie danych osobowych, w tym rozporządzenia Ministra Spraw Wewnętrznych i Administracji z dnia 29 kwietnia 2004 r. w sprawie dokumentacji przetwarzania danych osobowych oraz warunków technicznych i organizacyjnych, jakim powinny odpowiadać urządzenia i systemy informatyczne służące do przetwarzania danych osobowych (Dz. U. Nr 100, poz. 1024) oraz wytycznymi Generalnego Inspektora Ochrony Danych Osobowych. Dostęp do bazy mają jedynie osoby posiadające specjalne upoważnienia nadane przez Adrian Dyliński. \\
2. Adrian Dyliński prowadzi statystyki i badania dotyczące usług, ruchu w sieci, analizy finansowe. Mogą być one używane i wymieniane z podmiotami trzecimi, jednak żadne z tych statystyk nie zawierają jakichkolwiek danych osobowych Użytkowników Aplikacji. \\
3. Adrian Dyliński nie odpowiada za politykę prywatności stron internetowych, do których odnośniki umieszczone są na stronach serwisu.

\section{Oprogramowanie firm trzecich użyte w aplikacji \emph{Podręczna kucharyna} }
\subsection{MongoLab}
\emph{MongoLab} oferuje hosting baz danych z użyciem systemu zarządzania bazami danych \emph{MongoDB}.
Hosting jest dostępny w kilku wariantach cenowych.
Aplikacja \emph{Podręczna kucharyna} w wersji testowej korzysta z planu bezpłatnego.
Oferuje on do 250 MB na bazy danych.
Udostępnia także darmowy interfejs programowania aplikacji z wykorzystaniem wzorca projektowego REST. \\
Szczegóły wariantów cenowych można poznać pod adresem \url{"https://mongolab.com/about/products/#/pricing"}. \\
Warunki korzystania z usługi znajdują się pod adresem \url{"https://mongolab.com/legal/terms/"}\\
Polityka prywatności jest pod adresem \url{"https://mongolab.com/legal/privacy/"}
\subsection{HttpClient}
HttpClient jest zestawem klas ułatwiającycm programowanie aplikacji sieciowych.
Wchodzi on w skład biblioteki komponentów Apache Commons wydawanej przez Fundację Apache (Apache Software Foundation). \\
Warunki korzystania z oprogramowania znajdują się pod adresem \url{"http://www.apache.org/licenses/LICENSE-2.0"}.
\subsection{Jackson i bson4jackson}
\emph{Jackson} jest zestawem klas umożliwiającym łatwą obsługę formatu danuch JSON.
Jest to tekstowy format danych oparty na składni języka \emph{JavaScript}, szeroko rozpowszechniony w aplikacjach sieciowych ze względu na prostotę użytkowania. \\
\emph{Bson4Jackson} jest zestawem klas rozszerzających Jackson o możliwość obsługi formatu BSON.
Jest on binarnym zapisem formatu JSON i eliminuje jego główną wadę czyli rozmiar.\\
Oba zestawy klas wydawane są na zasadach licencji Apache w wersji 2.0.
Znajduje się ona pod adresem \url{"http://www.apache.org/licenses/LICENSE-2.0.txt"}.
\section{Oprogramowanie firm trzecich użyte podczas pisania i testowania aplikacji \emph{Podręczna kucharyna}}
\subsection{Eclipse IDE 3.7 (Indigo)}
Zintegrowane środowisko programistyczne zaprogramowane z użyciem biblioteki Eclipse. Składa się ono z szeregu dodatków wspomagających pisanie aplikacji w różnych językach, w tym w języku Java. Wszystkie dodatki używane przy pisaniu aplikacji \emph{Podręczna kucharyna} wdawane są na Publicznej Licencji Eclipse w wersji 1.0.
Dostępna jest ona pod adresem \url{"http://www.eclipse.org/legal/epl-v10.html"}
\subsection{Android\tm Software Development Kit i Android\tm Developer Tools}
Android\tm Developer Tools jest to zestaw narzędzi i dodatków do Eclipse IDE wspomagających programowanie aplikacji na system operacyjny Android\tm. Zawiera między innymi odpluskwiacz, emulator urządzeń mobilnych, program do zarządzania wersjami Android\tm Software Development Kit. \\
Android\tm Software Development Kit jest to zbiór komponentów i klas wykorzystywanych do programowania aplikacji na system operacyjny Android\tm. Jest on wydawany w kilkunastu wersjach. Każda z nich odpowiada jednej z wersji systemu operacyjnego Android\tm. Wyższe wersje zawierają więcej funkcji i są obsługiwane tylko przez nowsze urządzenia.
Aplikacja \emph{Podręczna kucharyna} korzysta z Android\tm SDK w wersji 10, co odpowiada wersji systemu operacyjnego 2.3.3. Urządzenia z nowszym systemem operacyjnym także są obsługiwane. \\
Warunki korzystania z Android\tm SDK i Android\tm Developer Tools można znaleźć pod adresem \url{"http://developer.android.com/legal.html"}
\subsection{JUnit}
Oprogramowania wspomagające pisanie testów jednostkowych. Udostępnia klasy i metody pozwalające w łatwy sposób opisać warunki wejściowe i wyjściowe danej metody. \\
Licencja i zasady użytkowania znajdują się pod adresem \url{"http://junit.sourceforge.net/cpl-v10.html"}.
Stronę projektu można znaleźć pod adresem \url{"http://junit.sourceforge.net/"}.
\subsection{mockwebserver}
Oprogramowanie wspomagające testowanie aplikacji sieciowych. Umożliwia zwrócenie z góry ustalonej odpowiedzi zamiast prawdziwego zasobu dostępnego przez sieć. Dzięki temu możliwe jest testowania aplikacji niezależnie od połączenia z siecią czy danych zwracanych przez wykorzystywane serwisy. \\
Warunki korzystania z oprogramowania znajdują się pod adresem \url{"http://www.apache.org/licenses/LICENSE-2.0"}. \\
Stronę projektu można znaleźć pod adresem \url{"http://code.google.com/p/mockwebserver/"}.

\end{document}
