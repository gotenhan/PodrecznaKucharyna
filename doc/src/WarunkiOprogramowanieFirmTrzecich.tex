\documentclass[12pt,leqno, twoside]{mwart}
\usepackage[utf8]{inputenc}
\usepackage{polski}
\usepackage{a4wide}
\usepackage{hyperref}
%-------------------------------------------------------------------------
%paginy!
\usepackage{fancyhdr}
\pagestyle{fancy}
\fancyhead{}
\renewcommand{\sectionmark}[1]{\markright{\thesection.\ #1 }}
\fancyhead[LE]{Podręczna kucharyna. Warunki korzystania i oprogramowanie firm trzecich}
\fancyhead[RO]{\rightmark}
\fancyfoot{} % clear all footer fields
\fancyfoot[LE,RO]{}
\fancyfoot[CE]{\thepage}
\fancyfoot[CO]{\thepage}
\addtolength{\headheight}{1.5pt} % pionowy odstep na kreske
\renewcommand{\headrulewidth}{0.4pt}
\renewcommand{\footrulewidth}{0.0pt}
%-------------------------------------------------------------------------
\makeatletter
\renewcommand{\@biblabel}[1]{\quad #1.}
\makeatother
%-------------------------------------------------------------------------
\renewcommand{\figurename}{Rys.}
\renewcommand{\labelitemi}{-}

\def\tm{\leavevmode\hbox{$\rm {}^{TM}$}}

\begin{document}

\thispagestyle{empty}
\begin{center}
Licencjacka Pracownia Oprogramowania \\ Instytut
Informatyki Uniwersytetu Wrocławskiego \\
\vspace{4cm}
\Large Agnieszka Góralczyk, Adrian Dyliński, Dominik Bruliński \\
\vspace{0.5cm}
\huge Dokumentacja projektu\\ \textbf{Podręczna kucharyna}\\ \Large Warunki korzystania z programu i oprogramowanie firm trzecich użyte w aplikacji \emph{Podręczna kucharyna} \\
\vspace{1cm}
\normalsize \today \\
\vspace{2cm}
\normalsize Wersja 0.01
\end{center}

\newpage

\section{Warunki korzystania z aplikacji \emph{Podręczna kucharyna} }
\section{Oprogramowanie firm trzecich użyte w aplikacji \emph{Podręczna kucharyna} }
\subsection{MongoLab}
\emph{MongoLab} oferuje hosting baz danych z użyciem systemu zarządzania bazami danych \emph{MongoDB}.
Hosting jest dostępny w kilku wariantach cenowych.
Aplikacja \emph{Podręczna kucharyna} w wersji testowej korzysta z planu bezpłatnego.
Oferuje on do 250 MB na bazy danych.
Udostępnia także darmowy interfejs programowania aplikacji z wykorzystaniem wzorca projektowego REST. \\
Szczegóły wariantów cenowych można poznać pod adresem \url{"https://mongolab.com/about/products/#/pricing"}. \\
Warunki korzystania z usługi znajdują się pod adresem \url{"https://mongolab.com/legal/terms/"}\\
Polityka prywatności jest pod adresem \url{"https://mongolab.com/legal/privacy/"}
\subsection{HttpClient}
HttpClient jest zestawem klas ułatwiającycm programowanie aplikacji sieciowych.
Wchodzi on w skład biblioteki komponentów Apache Commons wydawanej przez Fundację Apache (Apache Software Foundation). \\
Warunki korzystania z oprogramowania znajdują się pod adresem \url{"http://www.apache.org/licenses/LICENSE-2.0"}.
\subsection{Jackson i bson4jackson}
\emph{Jackson} jest zestawem klas umożliwiającym łatwą obsługę formatu danuch JSON.
Jest to tekstowy format danych oparty na składni języka \emph{JavaScript}, szeroko rozpowszechniony w aplikacjach sieciowych ze względu na prostotę użytkowania. \\
\emph{Bson4Jackson} jest zestawem klas rozszerzających Jackson o możliwość obsługi formatu BSON.
Jest on binarnym zapisem formatu JSON i eliminuje jego główną wadę czyli rozmiar.\\
Oba zestawy klas wydawane są na zasadach licencji Apache w wersji 2.0.
Znajduje się ona pod adresem \url{"http://www.apache.org/licenses/LICENSE-2.0.txt"}.
\section{Oprogramowanie firm trzecich użyte podczas pisania i testowania aplikacji \emph{Podręczna kucharyna}}
\subsection{Eclipse IDE 3.7 (Indigo)}
Zintegrowane środowisko programistyczne zaprogramowane z użyciem biblioteki Eclipse. Składa się ono z szeregu dodatków wspomagających pisanie aplikacji w różnych językach, w tym w języku Java. Wszystkie dodatki używane przy pisaniu aplikacji \emph{Podręczna kucharyna} wdawane są na Publicznej Licencji Eclipse w wersji 1.0.
Dostępna jest ona pod adresem \url{"http://www.eclipse.org/legal/epl-v10.html"}
\subsection{Android Software Development Kit i Android Developer Tools}
Android Developer Tools jest to zestaw narzędzi i dodatków do Eclipse IDE wspomagających programowanie aplikacji na system operacyjny Android\tm. Zawiera między innymi odpluskwiacz, emulator urządzeń mobilnych, program do zarządzania wersjami Android Software Development Kit. \\
Android Software Development Kit jest to zbiór komponentów i klas wykorzystywanych do programowania aplikacji na system operacyjny Android\tm. Jest on wydawany w kilkunastu wersjach. Każda z nich odpowiada jednej z wersji systemu operacyjnego Android\tm. Wyższe wersje zawierają więcej funkcji i są obsługiwane tylko przez nowsze urządzenia.
Aplikacja \emph{Podręczna kucharyna} korzysta z Android SDK w wersji 10, co odpowiada wersji systemu operacyjnego 2.3.3. Urządzenia z nowszym systemem operacyjnym także są obsługiwane. \\
Warunki korzystania z Android SDK i Android Developer Tools można znaleźć pod adresem \url{"http://developer.android.com/legal.html"}
\subsection{JUnit}
Oprogramowania wspomagające pisanie testów jednostkowych. Udostępnia klasy i metody pozwalające w łatwy sposób opisać warunki wejściowe i wyjściowe danej metody. \\
Licencja i zasady użytkowania znajdują się pod adresem \url{"http://junit.sourceforge.net/cpl-v10.html"}.
Stronę projektu można znaleźć pod adresem \url{"http://junit.sourceforge.net/"}.
\subsection{mockwebserver}
Oprogramowanie wspomagające testowanie aplikacji sieciowych. Umożliwia zwrócenie z góry ustalonej odpowiedzi zamiast prawdziwego zasobu dostępnego przez sieć. Dzięki temu możliwe jest testowania aplikacji niezależnie od połączenia z siecią czy danych zwracanych przez wykorzystywane serwisy. \\
Warunki korzystania z oprogramowania znajdują się pod adresem \url{"http://www.apache.org/licenses/LICENSE-2.0"}. \\
Stronę projektu można znaleźć pod adresem \url{"http://code.google.com/p/mockwebserver/"}.

\end{document}