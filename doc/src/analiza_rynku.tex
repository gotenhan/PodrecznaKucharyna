\documentclass[12pt,leqno, twoside]{mwart}
\usepackage[utf8]{inputenc}
\usepackage[polish]{babel}
\usepackage{polski}
\usepackage{a4wide}
%-------------------------------------------------------------------------
%paginy!
\usepackage{fancyhdr}
\pagestyle{fancy}
\fancyhead{}
\renewcommand{\sectionmark}[1]{\markright{\thesection.\ #1 }}
\fancyhead[LE]{Podręczna kucharyna. Analiza rynku}
\fancyhead[RO]{\rightmark}
\fancyfoot{} % clear all footer fields
\fancyfoot[LE,RO]{}
\fancyfoot[CE]{\thepage}
\fancyfoot[CO]{\thepage}
\addtolength{\headheight}{1.5pt} % pionowy odstep na kreske
\renewcommand{\headrulewidth}{0.4pt}
\renewcommand{\footrulewidth}{0.0pt}
%-------------------------------------------------------------------------
\makeatletter
\renewcommand{\@biblabel}[1]{\quad #1.}
\makeatother
%-------------------------------------------------------------------------
\renewcommand{\figurename}{Rys.}
\renewcommand{\labelitemi}{-}

\def\tm{\leavevmode\hbox{$\rm {}^{TM}$}}

\begin{document}

\thispagestyle{empty}
\begin{center}
Licencjacka Pracownia Oprogramowania \\ Instytut
Informatyki Uniwersytetu Wrocławskiego \\
\vspace{4cm}
\Large Agnieszka Góralczyk, Adrian Dyliński, Dominik Bruliński \\
\vspace{0.5cm}
\huge Dokumentacja projektu\\ \textbf{Podręczna kucharyna}\\ \Large Analiza rynku\\
\vspace{1cm}
\normalsize \today \\
\vspace{2cm}
\normalsize Wersja 0.02
\end{center}

\newpage


\begin{table}
	\centering
	\caption{Historia zmian w dokumencie}
		\begin{tabular}{|r|c|c|l|l|}
		\hline
		Lp. & Data       & Nr wersji & Autor               & Zmiana \\ \hline
		1.   & 2012-05-10 & 0.01 & Adrian Dyliński & Utworzenie dokumentu \\ \hline
		2.  & 2012-05-14 & 0.02 & Agnieszka Góralczyk & Poprawki językowe \\ \hline
		\end{tabular}
\end{table}

\newpage

\tableofcontents
\newpage
\section{Odbiorca aplikacji}
Aplikacja ,,Podręczna kucharyna'' skierowana jest do osób gotujących amatorsko lub zawodowo. Osoba taka gotuje na tyle często, że wyszukiwanie przepisów w~tradycyjnych książkach kucharskich lub spisanych na kartkach stało się uciążliwe. Odbiorcą programu są także osoby, które chcą mieć swoje ulubione przepisy zawsze przy sobie. ,,Podręczna kucharyna'' przydatna jest także dla osób na diecie ponieważ zawiera \textbf{kalkulator wartości odżywczych}.
\section{Konkurencyjne aplikacje}
Na rynku dostępnych jest już wiele aplikacji o~podobnym profilu. Część z~nich przeznaczona jest tylko dla komputerów stacjonarnych, część jest dostępna dla urządzeń przenośnych. Istnieją także inne rozwiązania adresujące problemy związane z~tradycyjnymi książkami kucharskimi. Poniżej znajduje się krótki opis wybranych rozwiązań:
\begin{description}
  \item[MasterCook] -- aplikacja komercyjna oferująca bardzo szeroki zakres możliwości: klasyfikowanie przepisów, zdjęcia, filmy, własne komentarze, listę ulubionych przepisów, wyszukiwanie po składnikach, \textbf{listę zakupów}, układanie menu, drukowanie, instruktaż głosowy, kalkulator wartości odżywczych. Zawiera także pokaźną bazę przepisów (ok. 8000). Master\nolinebreak[4]Cook jest dostępny na komputerach stacjonarnych i~urządzeniach przenośnych (tel. Apple iPhone), jego cena wynosi -- po przeliczeniu z~dolarów amerykańskich -- około 70 złotych.
  \item[Gourmet Recipe Manager] -- bezpłatna aplikacja na komputery stacjonarne. Oprócz podstawowych funkcji książki kucharskiej oferuje funkcję listy zakupów i~kalkulatora wartości odżywczych. Umożliwia dołączanie zdjęć do przepisów. Nie posiada własnej bazy przepisów za to ma funkcję importowania przepisów z~wielu innych programów i~witryn internetowych.
  \item[The BettyCrocker\textregistered\ CookBook for iPhone] -- bezpłatna aplikacja na telefony iPhone. Udostępnia podstawowe funkcje książki kucharskiej, takie jak wyszukiwanie (również po składnikach), listę ulubionych przepisów, zdjęcia potraw. Pozwala także przesłać przepis pocztą elektroniczną. Zawiera bazę około 9000 przepisów. Podczas korzystania z~aplikacji są wyświetlane reklamy.
\vfill\pagebreak
  \item[MyCookBook] -- aplikacja działająca pod kontrolą systemu operacyjnego Android\tm. Aktualny koszt to około 9~zł. Pozwala na dodawanie, redagowanie, importowanie i~wyszukiwanie przepisów. Przepisy mogą zawierać zdjęcia. Zgodna z~najpopularniejszymi formatami przepisów.
  \item[MediaChef\textregistered] -- urządzenie elektroniczne z~ekranem o~przekątnej 8 cali. Służy jako wideoksiążka kucharska (zawiera 48 przepisów), kalendarz, ramka na zdjęcia i~odtwarzacz mp3. Oprócz tego udostępnia przepisy w~formie tekstowej. Dołączony jest pilot. Filmy instruktażowe zostały nagrane z~udziałem zawodowego kucharza Briana Turnera.
  \item[Witryny internetowe] -- w~Sieci jest dostępnych mnóstwo witryn z~przepisami, zarówno amatorskich jak i~prowadzonych przez firmy z~branży spożywczej. Do tego dochodzą portale społecznościowe skupione wokół gotowania. Witryny te są ogromną bazą przepisów.
\end{description}
\section{,,Podręczna kucharyna'' na tle konkurencji}
Rynek oprogramowania książek kucharskich jest zróżnicowany. Z~jednej strony istnieją komercyjne aplikacje o~dużej liczbie funkcji i~wbudowanych bazach liczących kilka tysięcy przepisów, z~drugiej bezpłatne aplikacje o~ograniczonej funkcjonalności. ,,Podręczna kucharyna'' odpowiada dużym, komercyjnym aplikacjom dostępnym dla urządzeń mobilnych. Dzięki temu znacznie wybija się na tle aktualnie dostępnych rozwiązań dla urządzeń przenośnych, które są bardzo uproszczone. Dzięki możliwości importowania przepisów z~innych aplikacji liczba dostępnych w~,,Podręcznej kucharynie'' przepisów na potrawy i~koktajle jest praktycznie nieograniczona. ,,Podręczna kucharyna'' jest także atrakcyjna cenowo.

\end{document}
