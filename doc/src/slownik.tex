\documentclass[12pt,leqno, twoside]{mwart}
\usepackage[utf8]{inputenc}
\usepackage[polish]{babel}
\usepackage{polski}
\usepackage{a4wide}
%-------------------------------------------------------------------------
%paginy!
\usepackage{fancyhdr}
\pagestyle{fancy}
\fancyhead{}
\renewcommand{\sectionmark}[1]{\markright{\thesection.\ #1 }}
\fancyhead[LE]{Podręczna kucharyna. Słownik}
\fancyhead[RO]{\rightmark}
\fancyfoot{} % clear all footer fields
\fancyfoot[LE,RO]{}
\fancyfoot[CE]{\thepage}
\fancyfoot[CO]{\thepage}
\addtolength{\headheight}{1.5pt} % pionowy odstep na kreske
\renewcommand{\headrulewidth}{0.4pt}
\renewcommand{\footrulewidth}{0.0pt}
%-------------------------------------------------------------------------
\makeatletter
\renewcommand{\@biblabel}[1]{\quad #1.}
\makeatother
%-------------------------------------------------------------------------
\renewcommand{\figurename}{Rys.}
\renewcommand{\labelitemi}{-}

\def\tm{\leavevmode\hbox{$\rm {}^{TM}$}}

\begin{document}

\thispagestyle{empty}
\begin{center}
Licencjacka Pracownia Oprogramowania \\ Instytut
Informatyki Uniwersytetu Wrocławskiego \\
\vspace{4cm}
\Large Agnieszka Góralczyk, Adrian Dyliński, Dominik Bruliński \\
\vspace{0.5cm}
\huge Dokumentacja projektu\\ \textbf{Podręczna kucharyna}\\ \Large Słownik\\
\vspace{1cm}
\normalsize \today \\
\vspace{2cm}
\normalsize Wersja 0.06
\end{center}

\newpage

\begin{table}
	\centering
	\caption{Historia zmian w dokumencie}
		\begin{tabular}{|r|c|c|l|l|}
		\hline
		Lp. & Data       & Nr wersji & Autor               & Zmiana \\ \hline
		1.   & 2012-03-20 & 0.01 & Dominik Bruliński & Utworzenie dokumentu \\ \hline
		2. & 2012-04-17 & 0.02 & Agnieszka Góralczyk & Uzupełnienie treści \\ \hline
		3. & 2012-04-25 & 0.03 & Adrian Dyliński & Uzupełnienie treści \\ \hline
		4. & 2012-05-12 & 0.04 & Adrian Dyliński & Uzupełnienie treści \\ \hline
		5. & 2012-06-22 & 0.05 & Agnieszka Góralczyk & Uzupełnienie treści \\ \hline
		6. & 2012-06-22 & 0.06 & Dominik Bruliński & Uzupełnienie treści \\ \hline
		\end{tabular}
\end{table}

\newpage

\tableofcontents
\newpage

\section{Wstęp}
	Niniejszy dokument ma na celu prezentację definicji pojęć używanych w dokumentacji projektu \emph{Podręczna kucharyna}.
	\section{Słownik pojęć}
	
	\begin{description}
	
	\item[BSON] --- ang. Binary JSON, format wymiany danych używany głównie w MongoDB zdobywający coraz większą popularność
	
	\item[hosting] --- udostępnianie przez dostawcę usług internetowych zasobów serwerowni
	
	\item[interfejs użytkownika] --- ang. User Interface (UI), część oprogramowania zajmującą się obsługą urządzeń wejścia/wyjścia przeznaczonych do interakcji z użytkownikiem
	
	\item[JSON] --- ang. JavaScript Object Notation, format tekstowy będący podzbiorem języka JavaScript używany do wymiany danych
	
	\item[MVP] --- ang. Model–View–Presenter, architektoniczny wzorzec projektowy bazujący na MVC (Model–View–Controller). Zakłada on podział aplikacji na trzy główne części: model(reprezentacja logiki aplikacji), widok(wyświetlanie modelu za pomocą interfejsu użytkownika), prezenter(dwustronna komunikacja między modelem a widokiem)

	\item[REST] --- ang. Representational State Transfer, wzorzec architektury oprogramowania dla raplikacji klient - serwer zakłada bezstanowość, warstwowość, jednorodny interfejs, możliwość używania bufora podręcznego
	
	\item[test jednostkowy] --- ang. unit test, metoda testowania tworzonego oprogramowania poprzez wykonywanie testów weryfikujących poprawność działania pojedynczych elementów programu
	
	\item[testy akceptacyjne] --- testy, których celem nie jest wykrycie błędów a jedynie uzyskanie formalnego potwierdzenia wykonania oprogramowania odpowiedniej jakości
	
	\item[XML] -- ang. Extensible Markup Language, uniwersalny język formalny przeznaczony do reprezentowania różnych danych w strukturalizowany sposób
	
	\item[zintegrowane środowisko programistyczne] --- ang. Integrated Development Environment (IDE), aplikacja lub zespół aplikacji (środowisko) służących do tworzenia, modyfikowania, testowania i konserwacji oprogramowania
	
	\end{description}


\end{document}