\documentclass[12pt,leqno, twoside]{mwart}
\usepackage[utf8]{inputenc}
\usepackage[polish]{babel}
\usepackage{polski}
\usepackage{a4wide}
%-------------------------------------------------------------------------
%paginy!
\usepackage{fancyhdr}
\pagestyle{fancy}
\fancyhead{}
\renewcommand{\sectionmark}[1]{\markright{\thesection.\ #1 }}
\fancyhead[LE]{Podręczna kucharyna. Architektura aplikacji}
\fancyhead[RO]{\rightmark}
\fancyfoot{} % clear all footer fields
\fancyfoot[LE,RO]{}
\fancyfoot[CE]{\thepage}
\fancyfoot[CO]{\thepage}
\addtolength{\headheight}{1.5pt} % pionowy odstep na kreske
\renewcommand{\headrulewidth}{0.4pt}
\renewcommand{\footrulewidth}{0.0pt}
%-------------------------------------------------------------------------
\makeatletter
\renewcommand{\@biblabel}[1]{\quad #1.}
\makeatother
%-------------------------------------------------------------------------
\renewcommand{\figurename}{Rys.}
\renewcommand{\labelitemi}{-}

\begin{document}

\thispagestyle{empty}
\begin{center}
Licencjacka Pracownia Oprogramowania \\ Instytut
Informatyki Uniwersytetu Wrocławskiego \\
\vspace{4cm}
\Large Agnieszka Góralczyk, Adrian Dyliński, Dominik Bruliński \\
\vspace{0.5cm}
\huge Dokumentacja projektu\\ \textbf{Podręczna kucharyna}\\ \Large Architektura aplikacji\\
\vspace{1cm}
\normalsize \today \\
\vspace{2cm}
\normalsize Wersja 0.01
\end{center}

\newpage
\section{Wzorzec architektoniczny Model Widok Prezenter w aplikacji Podręczna kucharyna}
Aplikacja Podręczna kucharyna tworzona jest z wykorzystaniem wzoraca architektonicznego Model Widok Prezenter.
Dzięki temu w łatwy sposób można podzielić pracę na poszczegółnych uczestników projektu. 
\subsection{Opis wzorca MWP}
Wzorzec MWP zakłada podział aplikacji na trzy współpracujące ze sobą części:
\begin{itemize}
\item[Model] Część ta odpowiada za reprezentację danych w pamięci. Oprócz struktur przechowujących dane zawiera także metody pozwalające na nich operować: zmieniać właściwości, dodawać nowe i usuwać stare dane.
\item[Widok] Część odpowiedzialna za interakcję z użytkownikiem koncowym. Wyświetla dane w atrakcyjny wizualnie sposób, udostępnia interfejs, który umożliwia użytkownikowi dokonanie zmian w danych. Informuje pozostałe części programu o akcjach użytkownika.
\item[Prezenter] Odpowiada za komunikację między modułem modelu i widoku. Pobiera dane z modelu i powiadamia widok, że powinien uaktualnić to co jest przedstawiane użytkownikowi.
\end{itemize}
\end{document}
