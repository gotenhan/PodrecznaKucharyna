\documentclass[12pt,leqno, twoside]{mwart}
\usepackage[utf8]{inputenc}
\usepackage{polski}
\usepackage{a4wide}
%-------------------------------------------------------------------------
%paginy!
\usepackage{fancyhdr}
\pagestyle{fancy}
\fancyhead{}
\renewcommand{\sectionmark}[1]{\markright{\thesection.\ #1 }}
\fancyhead[LE]{Podręczna kucharyna. Testy}
\fancyhead[RO]{\rightmark}
\fancyfoot{} % clear all footer fields
\fancyfoot[LE,RO]{}
\fancyfoot[CE]{\thepage}
\fancyfoot[CO]{\thepage}
\addtolength{\headheight}{1.5pt} % pionowy odstep na kreske
\renewcommand{\headrulewidth}{0.4pt}
\renewcommand{\footrulewidth}{0.0pt}
%-------------------------------------------------------------------------
\makeatletter
\renewcommand{\@biblabel}[1]{\quad #1.}
\makeatother
%-------------------------------------------------------------------------
\renewcommand{\figurename}{Rys.}
\renewcommand{\labelitemi}{-}

\begin{document}

\thispagestyle{empty}
\begin{center}
Licencjacka Pracownia Oprogramowania \\ Instytut
Informatyki Uniwersytetu Wrocławskiego \\
\vspace{4cm}
\Large Agnieszka Góralczyk, Adrian Dyliński, Dominik Bruliński \\
\vspace{0.5cm}
\huge Dokumentacja projektu\\ \textbf{Podręczna kucharyna}\\ \Large Testy\\
\vspace{1cm}
\normalsize \today \\
\vspace{2cm}
\normalsize Wersja 0.01
\end{center}

\newpage


\begin{table}
	\centering
	\caption{Historia zmian w dokumencie}
		\begin{tabular}{|r|c|c|l|l|}
		\hline
		Lp. & Data       & Nr wersji & Autor               & Zmiana \\ \hline
		1.   & 2012-05-22 & 0.01 & Dominik Bruliński & Utworzenie dokumentu \\ \hline
		\end{tabular}
\end{table}

\newpage

\tableofcontents
\newpage

\section{Wstęp}
\noindent Dokument ten zawiera opis i~plan testowania oprogramowania Podręczna kucharyna w~każdej fazie powstawania tego
oprogramowania.

\section{Testy podczas implementacji}
\noindent Programista uczestniczący w projekcie odpowiada za dokładne przetestowanie pisanego przez siebie kodu
źródłowego. Testy powinny zostać wykonane przed umieszczeniem danego fragmentu w~magazynie systemu kontroli wersji.
Wybór między automatycznym i ręcznym sposobem wykonania tych testów pozostaje w~gestii programistów.

\section{Testowanie funkcjonalności}
\noindent Przed zakończeniem prac nad projektem Podręczna kucharyna zostaną wykonane testy, mające na celu sprawdzenie
poprawności implementacji poszczególnych przypadków użycia. Przypadki użycia są opisane w~\cite{APU}. Testy te zostaną wykonane
przez członków zespołu i~nie będą zautomatyzowane. Osoba testująca będzie wykonywała poszczególne przypadki użycia
,,krok po kroku''. Tester będzie sprawdzał, czy zachowanie oprogramowania odpowiada założeniom i~stawianym przed
oprogramowaniem Podręczna kucharyna wymaganiom. W celu przetestowania zachowania oprogramowania w sytuacjach
niepożądanych tester będzie postępował niezgodnie z przypadkami użycia. Zakłada się, że inforamacje o błędach,
niepowodzeniach wykonania operacji zostaną przekazane w postaci komunikatów. Każdy z~przypadków testowych jest opisany w
poniższych podrozdziałach.

\subsection{Uruchomienie aplikacji}
\subsubsection{Działania testujące}
\noindent Aplikacja jest uruchamiana zgodnie z~instrukcją. Następnie stan jest kontrolowany w~trakcie wszystkich innych testów.

\subsubsection{Sprawdzenie poprawności}
\noindent Po każdej interakcji użytkownika z interfejsem następuje sprawdzenie, czy stan aplikacji (polączenie, bazie
danych) uległ odpowiedniej zmianie. Sprawdza się, czy na przykład po imporcie przepisu z pliku bądź sieci Internet do
bazy danych został dodany poprawny wpis.

\subsection{Dodawanie przepisu, składnika}
\subsubsection{Działania testujące}
\noindent Otwarty zostaje ekran dodawania przepisu bądź składnika, a~następnie wykonywane są kroki opisane w~przypadkach
użycia.

\subsubsection{Sprawdzenie poprawności}
\noindent Kontrola działania polega na sprawdzeniu, czy został wyświetlony odpowiedni błąd w przypadku działań
niepożądanych oraz, czy stan aplikacji zmienił się odpowiednio w~przypadku działań, które powinny odnieść sukces.

\subsection{Wyszukiwanie przepisu, składnika}
\subsubsection{Działania testujące}
\noindent Otwarty zostaje ekran wyszukiwania, a~następnie wykonywane są kroki opisane w~przypadkach
użycia.

\subsubsection{Sprawdzenie poprawności}
\noindent Następuje sprawdzenie, czy uzyskane rezultaty wyszukiwania są zgodne z oczekiwaniami. Sprawdza się również, czy
w~przypadku, gdy parametry są niepoprawne, użytkownik zostaje powiadomiony stosownym komunikatem.

\subsection{Wyświetlanie przepisów, składników}
\subsubsection{Działania testujące}
\noindent Po otwarciu ekranu wyświetlania przepisów lub składników wykonywane są kroki opisane w~przypadkach użycia. 

\subsubsection{Sprawdzenie poprawności}
\noindent Sprawdzone zostaje czy wyświetlona została lista elementów z możliwością przejścia do widoku szczegółowego.
Sprawdza się również, czy w~przypadku szczególnym (np. brak przepisów) użytkownik zostaje powiadomiony stosownym
komunikatem.
\newpage
\begin{thebibliography}{9}
   \bibitem{APU} Agnieszka Góralczyk, Adrian Dyliński, Dominik Bruliński, {\it Dokumentacja projektu Podręczna
      kucharyna. Analiza przypadków użycia}. Wrocław, LPO IIUWr 2012.
   \bibitem{SLO} Agnieszka Góralczyk, Adrian Dyliński, Dominik Bruliński, {\it Dokumentacja projektu Podręczna
      kucharyna. Słownik}. Wrocław, LPO IIUWr 2012.
\end{thebibliography}


\end{document}
