\documentclass[a4paper,11pt,titlepage,twoside]{mwart}
\usepackage{polski}
\usepackage[utf8]{inputenc}
\usepackage[polish]{babel}

\frenchspacing
\linespread{1.2}
\usepackage{indentfirst}
\usepackage{a4wide}

\begin{document}

% ---- strona tytułowa
\begin{titlepage}
\begin{center}
\large{Instytut Informatyki Uniwersytetu Wrocławskiego\\Studencka Pracownia Inżynierii Oprogramowania}\\[3.0cm]
\Large {\bfseries Grupa G8:}\\
\large Adrian Dyliński, Aron Mojsak\\[4.0cm]
{ \Huge \bfseries Podręczna kucharyna}\\[0.5cm]
{ \large \bfseries Plan testowania}\\
\vfill
{\large Wrocław, dnia \today}
\end{center}
\end{titlepage}
% ---- pusta strona na odwrocie tytułowej
\mbox{}
\thispagestyle{empty}
\newpage
\setcounter{page}{1}
% ---- spis treści
\tableofcontents
\newpage
% ---- właściwa część
\section{Wstęp}
Dokument ten zawiera opis i~plan testowania aplikacji ,,Podręczna kucharyna''. W~dokumencie stosuje się zamiennie terminy aplikacja, program jako oznaczenia aplikacji ,,Podręczna kucharyna'', chyba, że z~kontekstu wyraźnie wynika inaczej. Wszystkie testy są przeprowadzane w~domyślnej konfiguracji. Każdy test przeprowadzany jest co najmniej dwukrotniej: w~,,czystym'' środowisku, (jest jedynym niesystemowym uruchomionym procesem) oraz w~systemie obciążonym innymi programami.
\section{Testy automatyczne}
Testy automatyczne przeprowadzane będą w~oparciu o~narzędzia testowania dla systemu Android przygotowane przez Google. 

W~ramach testów automatycznych będą przeprowadzane testy jednostkowe kodu. Każdy programista jest odpowiedzialny za przygotowanie testów jednostkowych do kodu przez siebie pisanego.

Wykonywane będą tzw. stress--tests interfejsu. Polega to na automatycznym ,,klikaniu'' w~elementy interfejsu, aby przetestować czy pod dużym obciążeniem aplikacja nie wyłącza się, nie występują naruszenia ochrony pamięci i~inne niechciane zachowania.

\section{Testy specyficzne dla systemu Android}
Ponieważ działanie systemu Android i~urządzeń mobilnych różni się znacznie od działania komputerów stacjonarnych, wykonane będą specyficzne testy. Każdy ekran aplikacji musi zostać przetestowany na reakcję na:
\begin{itemize}
  \item zmianę orientacji urządzenia,
  \item nieoczekiwane zakończenie aplikacji z~powodu braku pamięci, rozładowania baterii, zabicie procesu przez użytkownika za pomocą menedżera zadań itp.,
  \item opuszczenie aplikacji za pomocą wbudowanego w~urządzenie przycisku ,,Ekran główny'' i~późniejszy powrót do aplikacji
  \item zmiany w~konfiguracji urządzenia (np. zmianę języka, formatu wyświetlania dat itp.)
\end{itemize}
Dla każdego z~wyżej wymienionych przypadków sprawdzone zostanie czy aplikacja poprawnie powraca do stanu sprzed zajścia danego zdarzenia i czy aplikacja dalej działa tak jak zaplanowano. Dodatkowo testowany będzie wpływ działania aplikacji na żywotność baterii.
\section{Testy funkcjonalności}
Przeprowadzone zostaną test wszystkich funkcji programu. W~szczególności testowana będzie zgodność ze specyfikacją przypadków użycia. Sprawdzona zostanie poprawność operacji na bazie danych.

Importowanie przepisów będzie testowane półautomatycznie. Część testów oparta będzie na komputerowym porównianiu efektu importu specjalnie przygotowanych atrap przepisów z~oczekiwanymi wynikami, część będzie wykonywana w~całości przez testera. Dzięki temu będzie można zobaczyć jak aplikacja zachowuje się w~sytuacjach niestandardowych takich jak na przykład drobna niezgodność importowanego pliku ze specyfikacją formatu.

Na bieżąco będzie również sprawdzana łatwość użytkowania aplikacji. Testy te będą wykonywane przez osoby nie powiązane z~procesem twórczym. Na podstawie ich opinii będą wprowadzane zmiany w~interfejsie programu aby był on przyjazny dla użytkownika.

\end{document}
