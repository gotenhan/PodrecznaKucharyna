\documentclass[12pt,leqno, twoside]{mwart}
\usepackage[utf8]{inputenc}
\usepackage[polish]{babel}
\usepackage{polski}
\usepackage{a4wide}
%-------------------------------------------------------------------------
%paginy!
\usepackage{fancyhdr}
\pagestyle{fancy}
\fancyhead{}
\renewcommand{\sectionmark}[1]{\markright{\thesection.\ #1 }}
\fancyhead[LE]{Podręczna kucharyna. Testy}
\fancyhead[RO]{\rightmark}
\fancyfoot{} % clear all footer fields
\fancyfoot[LE,RO]{}
\fancyfoot[CE]{\thepage}
\fancyfoot[CO]{\thepage}
\addtolength{\headheight}{1.5pt} % pionowy odstep na kreske
\renewcommand{\headrulewidth}{0.4pt}
\renewcommand{\footrulewidth}{0.0pt}
%-------------------------------------------------------------------------
\makeatletter
\renewcommand{\@biblabel}[1]{\quad #1.}
\makeatother
%-------------------------------------------------------------------------
\renewcommand{\figurename}{Rys.}
\renewcommand{\labelitemi}{-}

\def\tm{\leavevmode\hbox{$\rm {}^{TM}$}}

\begin{document}

\thispagestyle{empty}
\begin{center}
Licencjacka Pracownia Oprogramowania \\ Instytut
Informatyki Uniwersytetu Wrocławskiego \\
\vspace{4cm}
\Large Agnieszka Góralczyk, Adrian Dyliński, Dominik Bruliński \\
\vspace{0.5cm}
\huge Dokumentacja projektu\\ \textbf{Podręczna kucharyna}\\ \Large Testy\\
\vspace{1cm}
\normalsize \today \\
\vspace{2cm}
\normalsize Wersja 0.01
\end{center}

\newpage


%trzeba zmieniać historię zmian, strone tytułowa i \fancyhead :)


\begin{table}
	\centering
	\caption{Historia zmian w dokumencie}
		\begin{tabular}{|r|c|c|l|l|}
		\hline
		Lp. & Data       & Nr wersji & Autor               & Zmiana \\ \hline
		1.   & 2012-04-10 & 0.01 & Dominik Bruliński & Utworzenie dokumentu \\ \hline
		2. & 2012-04-16 & 0.02 & Agnieszka Góralczyk & Uzupełnienie treści i poprawki językowe\\ \hline
		3 & 2012-06-21 & 0.03 & Dominik Bruliński & Uzupełnienie treści \\ \hline
		\end{tabular}
\end{table}

\newpage

\tableofcontents
\newpage

\end{document}